\documentclass{article}

\usepackage[left=3.5cm,right=3.5cm,top=3.5cm,bottom=3.5cm]{geometry}
\usepackage{graphics}
\usepackage{color}
\usepackage{fancyvrb}
\usepackage{textcomp}
\newcommand{\textapprox}{\raisebox{0.5ex}{\texttildelow}}
\usepackage{wrapfig}
\usepackage{enumitem}
\usepackage{amsmath}
\usepackage{geometry}
\usepackage{marginnote}

\newcommand{\topMarks}[1]{\marginnote{(#1 marks)}}

\newcommand{\solution}{\vspace{1em}\textcolor{red}{\textbf{Solution:}}}

\CustomVerbatimEnvironment{EVerbatim}{Verbatim}{
xleftmargin=0.5cm,
xrightmargin=0.2cm,commandchars=\\\{\},numbersep=0.9em,fontsize=\footnotesize}


\CustomVerbatimEnvironment{SVerbatim}{Verbatim}{formatcom=\color{blue},
xleftmargin=0.5cm,
xrightmargin=0.2cm,commandchars=\\\{\},numbersep=0.9em}

\date{}
\title{\vspace{-5em}CO545 Assessment 2 -- Adventure Game
{\footnotesize{(Brief v1.1)}}}

\begin{document}
\maketitle
\vspace{-5em}
\hrule
\vspace{1em}

\section{Overview}

In this assessment, you are going to make a text-based aventure game
in Haskell.\footnote{I am aware that your CO520 assessment was also a
  text adventure game. This provides a very useful comparison between
  the two styles/paradigms of programming. The particulars of the
  implementation and what is required of you will be quite different
  here. } The next three classes (week 17-19) will provide time
to work through each part of the assessment in chunks (but feel free
to jump ahead if you wish).

\textbf{This work must be your own}, but you may ask your class supervisor
for help, and they are able to give you hints as part of the class.

\paragraph{Formatting and submission instructions}
%
\begin{itemize}[itemsep=0.1em]
\item Download \texttt{Base.hs} from the Moodle which provides core
  data types and helpers.

\item You may modify the base if you wish for the last question of
  \textit{Chunk 3}, but please do the bulk of your work in a new file
  \texttt{Game.hs} which imports the base by including at the top:
%
\begin{align*}
\texttt{import Base}
\end{align*}
%

\item When submitting, create a zip file named after your user id,
  e.g. \texttt{dao7.zip}, containing your main code as
  \texttt{Game.hs} and including your \texttt{Base.hs} file;

\item Please limit your line lengths to 100 characters max.
\end{itemize}
%

\subsection{Base module}

This \texttt{Base.hs} module provides some key data types for
the game and some helper functions for outputting information
to the user:
%
\begin{itemize}
\item \texttt{GameState} -- holds a \texttt{Player} value and the
  current \texttt{Room} that the player is in;
%
\item \texttt{Next a} -- is used to represent the idea that,
after some command or action, either we stay the \texttt{Same} or
\texttt{Progress} to a new \texttt{a} (this will be used as \texttt{Next
  GameState});
%
\item \texttt{Item}s and \texttt{Monster}s;
\item \texttt{Player} -- contains information about the player
  including their inventory (a list of \texttt{Item});
\item \texttt{Room} -- describes a room and what is available
in it, including:
%
\begin{itemize}
\item The room's name, its description, and whether it is a ``winning
  room'' (ends the game) or not;
\item A \texttt{Maybe Item} representing whether an item is required to enter the room;
\item A list of \texttt{Item}s (each paired with a \texttt{String}
  description of where the item is located);
\item A list of \texttt{Monster}s;
\item An association list (list of pairs) between a \texttt{Direction}
  and a \texttt{Room};
\item A function representing actions possible in this room,
which takes an \texttt{Item} as input, a \texttt{GameState} as input,
and produces a \texttt{NextGameState}.
\end{itemize}
%
\item \texttt{tellContext} which takes a game state and outputs
the current room information.
\end{itemize}
%
Make sure you familiarise yourself with \texttt{Base.hs} and
understand each part of it.

\section*{Chunk 1 - Creating a map (30 marks)}

The following guides you to create a minimal map. You may create
something more elaborate (please not too elaborate!) but the following
components are required and will be marked.

Please feel free to chose the names and descriptions of rooms and the
positions of objects (any \texttt{String} piece of data is
your creative choice).

\begin{enumerate}
\item \topMarks{3} As a warmup (and a useful function for later)
  define a function \texttt{opposite} which computes a direction's
  opposite, e.g., \texttt{opposite North = South}. Include the type signature.

  \item \topMarks{3} In order to construct a \texttt{Room} an
  ``action'' value is needed which describes how a player can use an
  item in a room, possibly changing the game state. This is
  represented by function of type \texttt{Item -> GameState -> Next
    GameState}.

Define a simple such function which can be used
  for a room's \texttt{actions} field, of type:
%
\begin{align*}
\texttt{noActions :: Item -> GameState -> Next GameState}
\end{align*}
%
to represent a lack of actions in a room. Thus
\texttt{noActions} should return a \texttt{Same} value with a message that there is
nothing that can be done with the item given (you can use
\texttt{show} on an \texttt{Item}).

\item \topMarks{4} Define a winning room (has \texttt{isWinRoom} value as
  \texttt{True}) which requires a key to enter, has no items or
  monster or further doors, and no actions (use the function you
  defined above).

\item \topMarks{5} Define a starting room for your player from which the
  \texttt{winRoom} can be accessed and which has a spoon as one of its
  items (and no actions). This starting room should lead off to
  another room defined in the next question.

\item \topMarks{12} Define a room leading from the previous containing a wood troll with
  health \texttt{10} who holds a \texttt{Key}.  Implement an
  ``action'' function which on a \texttt{Spoon} input checks if the
  room (provided by the \texttt{GameState} parameter) has a monster:
 %
 \begin{itemize}
 \item If no monster, indicate (via \texttt{Same}) that we stay
   in the same state, giving an appropriate message.

  \item If there is a monster, we will attack it with the spoon (which
    deals 5 damage). If the monster has health less than or equal to 5
    before we attack then this action will kill it, so
    return a next game state where the monster is no longer present (not in the list) and
    the item it was \texttt{holding} is now put into the room's items.

    If the monster has $> 5$ health, update the monster's health in the
    room to be $5$ less.

    Include appropriate messages.
\end{itemize}

\item \topMarks{3} Define \texttt{game0} as an initial \texttt{GameState} pointing to your
start room and with a player who has an empty inventory.
\end{enumerate}

\newpage

\section*{Chunk 2 - Parsing commands and simple I/O (25 marks)}

The \texttt{Base} module provides the \texttt{Parsable} type class
with \texttt{parse :: String -> Maybe a}.

\begin{enumerate}[leftmargin=1.4em,resume]
\item Implement a \texttt{Parsable} instance for \texttt{Item}. Item
names should be parsed lower case.
%
\topMarks{3}

\item Implement a \texttt{Parsable} instance for \texttt{Direction}.
Directions should be parsed lower case.
%
\topMarks{3}

\item  \topMarks{12} Implement a \texttt{Parsable} instance for \texttt{Command}
  which parses the following inputs to commands:
%
\begin{itemize}[leftmargin=1em]
\item \texttt{go \emph{dir}} parsed to a \texttt{Move} (parsing also
the direction \texttt{\emph{dir}});

\item \texttt{grab \emph{item}} parsed to a \texttt{PickUp}
(parsing also the item \texttt{\emph{item}});

\item \texttt{use \emph{item}} parsed to a \texttt{Use} value
(parsing also the item \texttt{\emph{item}});

\item \texttt{end} parsed to \texttt{End}.
%
\end{itemize}
\emph{Hint: remember strings are lists and you can pattern match on a
  number of characters at the start of a list by nested cons patterns.}

%
\item \topMarks{3} Define a function \texttt{tellResponse}
that takes a string and outputs it (i.e., writes to standard output)
with the following form, e.g., where \texttt{message} is the input string here:
%
\begin{align*}
\texttt{< message .}
\end{align*}
%
\emph{Hint: check \texttt{tellContextLine} from \texttt{Base} for inspiration.}

\item \topMarks{4} Define a function \texttt{readCommand :: IO (Maybe Command)} that
outputs the string \texttt{"> "} using
\texttt{putStr}\footnote{Depending on your OS you may also need to
  evaluate \texttt{hFlush stdout} (imported from \texttt{System.IO})
after \texttt{putStr} for a non-new-line string to appear, particularly when
running the compiled version later.}
and then reads a line of input from the user with \texttt{getLine}
and returns the result of parsing the user's input string via
\texttt{parse}.
%
\end{enumerate}

\section*{Chunk 3 - Game engine (45 marks)}

\begin{enumerate}[leftmargin=1.4em,resume]

\item \topMarks{5} As part of the game engine, items are deleted from
  a room when they are picked up. Items are stored in an association
  list (list of pairs), therefore define a function of the type:
%
\begin{align*}
\texttt{deleteFrom :: Eq a => a -> [(a, b)] -> [(a, b)]}
\end{align*}
%
which, from the second parameter, deletes the pair whose first
component is equal to the first parameter.

\item \topMarks{4} During the course of the game engine, a room may
be recomputed (e.g., items removed, monster wounded
or killed). When we move from one room to another, we want
the new room to point back to the previous room we were in. Define
a function of type:
%
\begin{align*}
\texttt{leaveRoom :: Room -> Direction -> Room -> Room}
\end{align*}
%
where \texttt{leaveRoom fromRoom dir toRoom} returns \texttt{toRoom}
but whose \texttt{doors} have been updated such that the opposite
direction is now associated to \texttt{fromRoom}.

\emph{Hint: use \texttt{opposite} and
\texttt{deleteFrom}.} You will be able to test your function
more easily later so you might want to come back to this after
writing something that type checks.


\item \topMarks{15} Implement a function \texttt{step} which takes a command,
a game state, and computes what to do next, i.e., of type:
%
\begin{align*}
\texttt{step :: Command -> GameState -> Next GameState}
\end{align*}
%
This function should handle each possible
command as follows:
%
\begin{enumerate}
% 8 marks
\item \texttt{Move dir} should check
if there is a door in that direction and if so, and if the player's
inventory has the required item for that door, should return
a new game state (\texttt{Progress}) where the current room of game
state is that new room (computed using \texttt{leaveRoom})

\emph{Hint: you may want to use the \texttt{lookup} function to
lookup the door from the} \texttt{doors} \emph{association list.}

%  4 marks
\item \texttt{PickUp item} should
remove an item from the room (\texttt{deleteFrom}) and add it to the player's
inventory if indeed the item is in the room.

Include appropriate messages, considering also the case where
the requested item is not in this room.

% 3 marks
\item \texttt{Use item}, if the item
is in the player's inventory, then
apply the current room's \texttt{actions} function
to the item to compute the next game state.

Otherwise, stay in the same state with a message
that the player doesn't have the item.
\end{enumerate}

\item \topMarks{12} Implement the functions:
%
\begin{align*}
& \texttt{play :: GameState -> IO ()} \\
& \texttt{playLoop :: GameState -> IO ()}
\end{align*}
%
which should have the following behaviour:
%
\begin{itemize}
\item \texttt{play} should simply report the context of the player based on the current game state
  (using \texttt{tellContext}) and then apply \texttt{playLoop}.

\emph{Hint: use sequencing of \texttt{IO} computations via either
the \emph{do} notation or \texttt{>>=}}
%
\item \texttt{playLoop} should first check if the player is in the winning room
  and if so end the game;
\item Otherwise, read an input from the user with \texttt{getCommand}, and:
 \begin{itemize}[leftmargin=1em]
   \item If the command is unknown, tell the user and recurse on
     \texttt{playLoop} with the current game state;
   \item If the command is \texttt{End}, return after outputting an
     appropriate message.
   \item Otherwise, use the \texttt{step} function to determine
     the next game state. If \texttt{Same}, show the
     message and recurse with the current game state; if
     \texttt{Progress}, show the message, \texttt{tellContext} for the
     new game state, and recurse with this new game state.
\end{itemize}
\end{itemize}

\item \topMarks{1} Define the \texttt{main} function (to make your
code compilable as a stand-alone executable) which starts the game
with the \texttt{game0} state (see Chunk 1 q6).

\item \topMarks{8} \textbf{Bonus} Up to a maximum of 8 bonus marks for
adding some other feature, e.g., one of:
%
\begin{itemize}
\item A new item and some action demonstrating it;
\item A new monster and action demonstrating it;
\item A help menu;
\item Player health / monster attacks.
\end{itemize}
%
\end{enumerate}
%

\newpage

\appendix

\section{Example game play from my minimal model answer}

\begin{EVerbatim}
   You are in a kitchen. It is cold, dark.
   There are doors to the south and east.
   On a table there is a spoon.

> go east
< This door requires a key.
> use spoon
< You do not have a spoon.
> grab spoon
< You picked up the spoon.

   You are in a kitchen. It is cold, dark.
   There are doors to the south and east.

> go south
< You go through the door to the south.

   You are in a garden. It is weedy.
   There is a door to the north.
   There is a wood troll holding a key.

> use spoon
< You attack the monster but it seems only wounded.

   You are in a garden. It is weedy.
   There is a door to the north.
   There is a wood troll holding a key.

> use spoon
< You attack the monster with the spoon and it succumbs. It drops a key.

   You are in a garden. It is weedy.
   There is a door to the north.
   On the floor there is a key.

> grab key
< You picked up the key.

   You are in a garden. It is weedy.
   There is a door to the north.

> go noerg
Unknown command
> go north
< You go through the door to the north.

   You are in a kitchen. It is cold, dark.
   There are doors to the south and east.

> go east
< You unlock the door to the east.

   You are in a loot room. It is shiny. You get all the loot..
   There is a door to the west.

< You won Dominic! Well done.
\end{EVerbatim}

\end{document}